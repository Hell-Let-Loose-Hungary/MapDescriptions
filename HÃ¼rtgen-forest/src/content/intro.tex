\chapter{Bevezetés}
A Hürtgen-erdő Belgium és Németország határán húzódik, középen a Siegfried vonal védelmi erődítményei szelik ketté a területet.

A fás területeken való hard elszórtan előrodul, az alacsony látási viszonyok miatt az egységeknek keedvezőtlen összecsapásokba kell belemenniük.

A sűrű lombok és kanyargós utak miatt a páncélos haderőnek nehéz betölteni a szerepét. Az erdei utakon elhelyezett aknák miatt fokozott óvatossággal kell haladniuk.

A gyalogság ebből a szempontból könnyebben boldogul, a fák által biztosított fedezéket kihasználva tudnak támadni célpontokat.